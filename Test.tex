%Beginning of sample document

\documentclass[draft]{article}
\usepackage{amsmath}
\usepackage{amsfonts}
\usepackage{amssymb}
\begin{document}
\begin{flushright}
\today
\end{flushright}
\textbf{A Note on Limit Statements in Standard and Nonstandard Analysis}
\\[10pt]
In analysis it is common to consider some notion of approximating small quantities, either via geometric tangents in which concepts as derivatives, differentials, and increments follow or in the algebraic notion in which limits, continuity, cardinality of infinite sets, and completeness are found. However, this notion of a limiting process in standard analysis is cumbersome and its rigor was only found by Cauchy via epsilon-delta proofs and the notion of topological neighborhoods. Compare this notion to non-standard analysis which is far more intuitive, and gets to heart of the matter via consider hyperreal numbers. 
\begin{center}
[

$$ \lim$$ $$(h \to 0)$$ $$\frac{\sin(x+h)-\sin(x)}{h}$$

]
\end{center}
While clearly a difference quotient this use of the limit on h is confusing especially on $\infty$ process.
Compare this to
\begin{center}
Given $$\mathbb{R*}$$ define: f'(x)= stp($$\frac{f(x+s)-f(x)}{s}$$)
\end{center}
Thus it becomes clear that the middle terms containing s go away as $$s$$ $$\mathtt{e}$$ $$\mathbb{R}$$

\end{document}


