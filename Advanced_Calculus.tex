\documentclass{article}
\usepackage{amsfonts}
\title{Advanced Calculus- Notes}
\author{Ibraheem Khan}
\begin{document}
\maketitle{}
\section{Introduction}
This is a text in production. Check back for updates. This text will compile a series of notes taken regarding various methods of analysis and in general will attempt to develop an interesting range of externalities related to the traditional Calculus sequence. Problems and proofs will be included. In addition, this text will serve as practice for LaTeX typesetting. Images will be handled in Matlab and Postcript (particularly for fractals).
Topics covered will include:
\begin{itemize}
 \item The bivalence of strong and weak induction via the well-ordering-principle of $\mathbb{N}$
  \item Solution formulae for low-degree polynomials and the impossibility on the quintic
  \item Euler's identity, complex numbers, and the symmetry of circular and hyperbolic functions
  \item Elliptic functions and Elliptic integrals
  \item Proofs of Sandwich Theorem
  \item Proofs of Inverse Function Theorem
  \item Topics in Mean Value Theorems and Cauchy's MVT, Rolle's Theorem
  \item Brief Differential Geometry
  \item Generalized L'Hopitals Rule
  \item A 'Discrete' L'Hopitals- Stolz-Cesaro Theorem
  \item Cersaro Summation and other strange things
  \item Error in Standard Linear Approximation
  \item Some Non-Standard Analysis and Increment Theorem
  \item Generalizations of Pappus's Three Theorems and Parallel Axis Theorem
  \item Advanced Integration Techniques
  \item Proof of Generalized Chain Rule
  \item Reynolds Transport Theorem
  \item Jacobians
  \item Interlude into Differential Forms, Manifolds, and Helmholtz's Theorem
  \item Multivariable implicit differentiation
  \item Generalized Stokes Theorem
  \item Ostrogradsky's Method
  \item Newton-Rhapson
  \item Osborn's Rule
  \item Cauchy's Formula for Repeated Integration
  \item Continuity and Holder Conditions
  \item Irrotationality
  \item Centroid of a Solid Semiellipsoid using Generalized Pappus's Centroid Theorem
  \item Gamma Functions
  \item Stirling Approximations
  \item Hamiltonian's and Laplacians
\end{itemize}
\end{document}
